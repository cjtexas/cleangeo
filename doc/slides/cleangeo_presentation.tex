\documentclass{beamer}
  \usepackage[utf8]{inputenc}
  \usepackage{hyperref}
  \usepackage{xspace}
  \usepackage{xcolor}
  \usepackage{graphicx}
  \usepackage{Sweave}
  \usepackage{multicol}
  
  \definecolor{links}{HTML}{2A1B81}
  \hypersetup{colorlinks,linkcolor=,urlcolor=links}  
  
  %Beamer properties
  \usetheme{Warsaw} %{Boadilla}
  \useoutertheme{miniframes}
  \usecolortheme{seahorse}
  \usecolortheme{rose}
  \setbeamerfont{footnote}{size=\tiny}

  % Add R Logo
  \definecolor{Rcolor}{RGB}{150,160,190}

  \newcommand{\Rx}{\fontsize{10pt}{12pt}\selectfont
  	\raisebox{.3em}{\hspace{1.2em}%
  		\llap{\resizebox{1.09em}{.5em}{\color{black}$\bigcirc$}}%
  		\llap{\resizebox{1.199em}{.55em}{\color{darkgray}$\bigcirc$}}%
  		\llap{\resizebox{1.19em}{.52em}{\color{gray!50}$\bigcirc$}}%
		\llap{\resizebox{1.1em}{.5em}{\color{gray}$\bigcirc$}}%
		\llap{\resizebox{1.25em}{.55em}{\color{gray}$\bigcirc$}}%
	}%
	\hspace{-.85em}%
	\textbf{%
		%\resizebox{.55em}{1.5ex}{\textcolor{black!80}{\textsf{R}}}%
		\textcolor{black}{\textsf{R}}%
		\hspace{-.025em}\raisebox{.01em}{\llap{\textcolor{Rcolor}{\textsf{R}}}}%
	}}%

  \newbox\rbox
  \savebox\rbox{\scalebox{0.1}{\Rx}}
  \makeatletter
  \def\R{\scalebox{\f@size}{\usebox\rbox}\xspace}
  \makeatletter
  
  %S4 representations (in case)
  \newenvironment<>{cleangeo}[1]{%
  \DefineVerbatimEnvironment{Soutput}{Verbatim}{fontsize=\tiny,formatcom=\color{black}}
  \setbeamerfont{block title}{size=\normalsize}
  \setbeamerfont{block body}{size=\scriptsize}
  \setbeamercolor{block title}{fg=white,bg=gray!90!black}%
  \begin{block}#2{#1 \R representation}}{\end{block}}
  
  %R code
  \DefineVerbatimEnvironment{Sinput}{Verbatim}{fontsize=\scriptsize,formatcom=\color{blue}}
  \DefineVerbatimEnvironment{Soutput}{Verbatim}{fontsize=\tiny,formatcom=\color{gray}}

  %cleangeo presentation%
  %---------------------%

  \title{cleangeo - Cleaning Geometries from Spatial Objects in \R}
  \author[E. Blondel]{Emmanuel Blondel \\ \texttt{emmanuel.blondel1@gmail.com}}
  \institute{CEO - International Consultant}
  \date{September 30, 2015}

\AtBeginSection[]
{
  \begin{frame}<beamer>
    \frametitle{Outline}
    \tableofcontents[currentsection]
  \end{frame}
}

%\setbeamertemplate{section in toc}{\protect\block{Research Question~\inserttocsectionnumber}\inserttocsection\protect\endblock}

\begin{document}

  \begin{frame}
  	\titlepage
  \end{frame}

\section{Introduction}
  
  %Introducing cleangeo%
  %--------------------%
  \begin{frame}
  	\frametitle{cleangeo}
  	\framesubtitle{R spatial data processing}
  	
	\R provides a lot of facilities to read and manipulate spatial data. We can mention some packages such as:
	\begin{itemize}
		\item \texttt{sp}, which provides a set of classes and methods to handle spatial objects
		\item \texttt{maptools}, a set of tools for reading and handling Spatial Objects
		\item \texttt{rgdal}, which provides an interface to the well-known \href{http://www.gdal.org/}{Geospatial Data Abstraction Library (GDAL)}
		\item \texttt{rgeos}, which provides an interface to the widely used \href{http://trac.osgeo.org/geos/}{Geometry Engine, Open Source (GEOS)} library. 	
	\end{itemize}	  	
  	
	The main problem comes when spatial objects did not have valid geometries, or expose different types of geometry errors (e.g. orphaned holes), which prevents any easy spatial data processing.  	
  	
  \end{frame}

  %cleangeo package%
  %-------------%
  \begin{frame}
  	\frametitle{cleangeo}
  	\framesubtitle{A package for cleaning geometries from spatial objects in R}
  	
	\texttt{cleangeo} was built in order:
	\begin{itemize}
		\item to facilitate handling and catching rgeos geometry issues
		\item to provide an utility to clean the spatial objects.
	\end{itemize}

\vspace{1cm}
	 	
	\texttt{cleangeo} appears to a useful tool to check, and eventual clean the data prior any spatial data processing.	
  	
  \end{frame}
  
 
\section{Using cleangeo} 
 
  %cleangeo package - Usage %
  %----------------------%
  %Installing cleangeo%
  \begin{frame}[containsverbatim]
  	\frametitle{cleangeo}
  	\framesubtitle{Usage - Installing cleangeo}
  	\texttt{cleangeo} can be installed:
  	\begin{itemize}
  		\item from CRAN
\begin{Schunk}
\begin{Sinput}
R> install.packages("cleangeo")
\end{Sinput}
\end{Schunk}
 	 
 	 	\vspace{0.1cm}
  		\item from Github, for latest updates (requires \texttt{devtools} package)
\begin{Schunk}
\begin{Sinput}
R> require(devtools)
R> install_github("eblondel/cleangeo")
\end{Sinput}
\end{Schunk}
  	\end{itemize}
	
	\vspace{0.1cm}
	Load \texttt{cleangeo} in R using:
\begin{Schunk}
\begin{Sinput}
     R> require(cleangeo)
\end{Sinput}
\end{Schunk}
		 
  \end{frame}
 
  %Usage - Identify geometry issues in spatial objects%
  \begin{frame}[containsverbatim]
  	\frametitle{cleangeo}
  	\framesubtitle{Usage - Identify geometry issues in spatial objects}

Let's load some sample data in \R:

\begin{Schunk}
\begin{Sinput}
R> require(maptools)
R> file <- system.file("extdata", "example.shp", package = "cleangeo")
R> sp <- readShapePoly(file)
\end{Sinput}
\end{Schunk}

\texttt{cleangeo} first allows to inspect spatial objects, and detect eventual issues:

\begin{Schunk}
\begin{Sinput}
R> sp.report <- clgeo_CollectionReport(sp)
R> sp.summary <- clgeo_SummaryReport(sp.report)
\end{Sinput}
\begin{Soutput}
             type     valid                 issue_type
 rgeos_validity:2   Mode :logical   GEOM_VALIDITY:2   
 NA's          :1   FALSE:2         NA's         :1   
                    TRUE :1                           
                    NA's :0                           
\end{Soutput}
\end{Schunk}

The output indicates that 2 spatial objects have a problem of geometry validity.

  \end{frame}
  
 %Usage - Clean geometries spatial objects%
  \begin{frame}[containsverbatim]
  	\frametitle{cleangeo}
  	\framesubtitle{Usage - Clean geometries spatial objects}

\texttt{cleangeo} allows you to clean spatial objects:

\begin{Schunk}
\begin{Sinput}
R> sp.fixed <- clgeo_Clean(sp)
\end{Sinput}
\end{Schunk}

Let's analyse the new spatial objects:

\begin{Schunk}
\begin{Sinput}
R> sp.fixed.report <- clgeo_CollectionReport(sp.fixed)
R> sp.fixed.summary <- clgeo_SummaryReport(sp.fixed.report)
\end{Sinput}
\begin{Soutput}
   type    valid         issue_type
 NA's:3   Mode:logical   NA's:3    
          TRUE:3                   
          NA's:0                   
          
\end{Soutput}
\end{Schunk}

And that's it! Now your spatial objects have clean geometries, you can safely do some spatial data processing on it. 

  \end{frame}
  
  
\section{Conclusions \& Perspectives}
  
  %rsdmx package - Conclusions & Perspectives %
  %-------------------------------------------%
  \begin{frame}
  	\frametitle{cleangeo}
  	\framesubtitle{Conclusions and Perspectives}
  	
Although \texttt{cleangeo} is at its early stages, it has proved to be a real benefit in the field of spatial data processing, providing a \textbf{quick} and \textbf{simple} way to check geometry issues and clean spatial objects.
  		
\vspace{1cm} 		
  		
\texttt{cleangeo} is currently focused on cleaning \texttt{SpatialPolygons} objects. A direct perspective is to extend it to other \texttt{Spatial*} objects.
 
\vspace{1cm}  
  		
Any suggestion from users is welcome to strenghten the \texttt{cleangeo} package.
  	
  \end{frame}
  
   %rsdmx package - Fundings %
  %--------------------------%
  \begin{frame}
  	\frametitle{cleangeo} 
   	\framesubtitle{Looking for Sponsors}	
  	
\texttt{cleangeo} was initially born from some assistance provided to users that were facing issues in processing spatial data in R (see the original \href{http://gis.stackexchange.com/questions/113964/fixing-orphaned-holes-in-r}{post}). The prototype package was made available on Github on a voluntary basis in support to this users request.

\vspace{0.5cm}

To guarantee the sustainability of \texttt{cleangeo}, we are seeking for \textbf{fundings}, throught \textit{sponsoring} or \textit{donations} to:
  	\begin{itemize}
		\item \textbf{implement, test, validate and release improvements}
		\item \textbf{guarantee a quality maintenance} of the package
		\item \textbf{provide support} to users and institutions that may take advantage of \texttt{cleangeo}
  	\end{itemize}

\vspace{0.5cm}  	
  	
If you are interesting in supporting \texttt{cleangeo}, please do not hesitate to contact \href{mailto:emmanuel.blondel1@gmail.com}{me}!  	
  	
  \end{frame}
  
  %cleangeo package - cleangeo on the web %
  %---------------------------------------%
  \begin{frame}
  	\frametitle{cleangeo}
  	\framesubtitle{on the web}
  	\begin{itemize}
		\item on Github:
		\begin{itemize}
			\item source code: \href{https://github.com/eblondel/cleangeo}{https://github.com/eblondel/cleangeo}
			\vspace{0.1cm}
			\item online documentation: \href{https://github.com/eblondel/cleangeo/blob/master/vignettes/quickstart.Rmd}{https://github.com/eblondel/cleangeo/blob/master/vignettes/quickstart.Rmd}
			\vspace{0.1cm}
		\end{itemize}	
		\vspace{0.1cm}
		\item on the Comprehensive R Archive Network (CRAN): \href{http://cran.r-project.org/package=cleangeo}{http://cran.r-project.org/package=cleangeo}
  	\end{itemize}  	
  \end{frame}  
  
  

\end{document}